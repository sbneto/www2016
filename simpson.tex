\section{Importance of cohorts: Puzzling outcomes}
From the prior analysis, cohorts emerge as an important factor when analyzing activity on a community like Reddit. Because of this heterogeneity, any analysis that speaks of users' activity should account for cohorts and perhaps other kinds of heteregeneity in a community's participants. In this section, we show that accounting for cohorts is not only a desirable property, but a vital one: not doing so can lead to analysis with absolutely wrong conclusions. 

Let us consider Figure x, which shows the average comment size on Reddit over time. We see a clear trend towards declining sizes of comments. Across all users, we see that average size of a comment decreases as the community grows older. This could be a warning sign for reddit community managers, assuming longer comments are associated with more involved users and healthier discussions. A data analyst looking at these numbers might think about ways to incentivize or promote longer comments on Reddit. 

However, in Figure 7, we saw that average comment size increases over time for different cohorts. While later cohorts start at smaller comment sizes, all of the cohorts show a positive trend towards writing bigger comments as time goes on. This is puzzling: when each of the cohorts exhibit a steady increase in their average comment size, how can the overall mean comment size decrease? This anomaly is an instance of the Simpson's paradox, and occurs because we fail to properly condition on different cohorts when computing mean comment length. 

%Sam 9: Still unclear if we should calculate the proportion above and bellow the median of the previous year to have a ``percentage like'' number just as in the wikipedia example. I would like to have this set up as similar as possible as the easiest reference people can find about the subject. A little bit going up by the end, although for some analysis including 2014 and 2007 is a problem, I don't necessarily think it gets in the way here, should we get rid of them?
\begin{table}[htbp]
\centering
\tabcolsep=0.11cm
\singlespacing
\fontsize{7pt}{8pt}\selectfont
\begin{tabular}{|>{\raggedright\centering\arraybackslash}m{1.5cm}|c|}
\hline
Year & Median \\ \hline
2007 & 114 \\ \hline
2008 & 103 \\ \hline
2009 & 103 \\ \hline
2010 & 96 \\ \hline
2011 & 91 \\ \hline
2012 & 89 \\ \hline
2013 & 87 \\ \hline
2014 & 88 \\ \hline
\end{tabular}
\caption{Evolution of the median throughout the years for the whole reddit dataset.}
\end{table}


\begin{table}[htbp]
\centering
\tabcolsep=0.11cm
\singlespacing
\fontsize{7pt}{8pt}\selectfont
\begin{tabular}{|>{\raggedright\centering\arraybackslash}m{1.5cm}|c|c|c|c|c|c|c|c|}
\hline
Year & 2007 & 2008 & 2009 & 2010 & 2011 & 2012 & 2013 & 2014 \\ \hline
2007 & 114 & - & - & - & - & - & - & - \\ \hline
2008 & 106 & 99 & - & - & - & - & - & - \\ \hline
2009 & 113 & 101 & 99 & - & - & - & - & - \\ \hline
2010 & 114 & 103 & 96 & 91 & - & - & - & - \\ \hline
2011 & 119 & 109 & 103 & 93 & 83 & - & - & - \\ \hline
2012 & 125 & 114 & 110 & 101 & 87 & 81 & - & - \\ \hline
2013 & 126 & 117 & 111 & 104 & 92 & 82 & 80 & - \\ \hline
2014 & 128 & 119 & 113 & 106 & 95 & 87 & 83 & 82 \\ \hline
\end{tabular}
\caption{Evolution of the median throughout the years for each cohort. Each column here is one cohort and each line is one year in time. Cohorts only start having data on the cohort year, therefore the upper diagonal is blank.}
\end{table}

%Sam 9: Using medians instead of average as Amit suggested, gives a better idea that ``more than 50% of the comments of the next year are smaller than the previous year''
Table x provides some clues to what might be going on. For illustration, we consider the change in average comment length from the year 2011 to 2012. Overall, comment length is increasing. If all users had similar average comment lengths, then we would also see that average length across cohorts is decreasing with time. However, people in later cohorts tend to write less per comment. Since their numbers increase year by year, we have a  much larger contribution from them towards comments, compared to users of earlier cohorts. This uneven contribution leads to the paradox we observe in the Table. 

With the decision to condition on cohorts, one may have gathered an entirely wrong conclusion. People  are not starting to write less, rather those who tend to write less are joining the  added to the community. Knowing this, one may focus on better onboarding processes for newcomers, or evaluate why users in later cohorts tend to write smaller comments on average. 