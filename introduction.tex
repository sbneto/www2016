\section{Introduction}

In previous work, researchers have studied the relationship of different cohorts adopting new technologies and how users that did not grow in a technological environment show different characteristics when compared with the younger generations. This external variable to the social network might explain many different aspects of how adoption of a network happens. Just as users experience outside the network vary according to their age and influence their behavior, users' experience inside the network throughout time vary as the network evolves. Users in the early stages of a social network have a very different experience from latter users.

Are users evolving in different ways based on when they join the network? How is an early users different from a latte user?
What can ``different'' be? Effort, activity, survival?
Are communities evolving in different ways based on when they join the network?
Is there a consolidation point in a social network where the ``core content'' is established? Can this core change over time?
Are latter users intrinsically different from earlier users or are they having different initial experiences?

User-Network homophily? They connect because they are similar or do they become similar as the user evolves? Are the ``dissimilar'' leaving?
Looking at how reddit looked like at a particular point in time is a different question from how users evolve, and much of the user evolution depends on the environment a user finds when they first join the network. In many ways, this is an initial value problem, but separating what is due to the evolution of the network and what comes from the different demographics outside the network is not always clear.
