%% DC 12: I think we should narrow the story and focus it on user effort.  We can define things like the joint subreddit/user cohort analysis as a kind of effort question, but we need to have a story that's narrow enough to tell in a believable way and that people can hang their hats on.  That's what I tried to do here.

\section{Introduction}

Understanding the evolution of users in a social network is essential for a variety of tasks: monitoring community health, predicting individual user trajectories, and supporting effective recommendations, among others.  Many works aim at explaining these temporal aspects of evolution. Some adopt a point of view of the whole network and try to understand the patterns of behavior inside of the network \cite{Zhu2014, Kooti2010}, while others adopt a more user-centric point of view and try to model \cite{Correa2010, Priedhorsky2007, Panciera2009, Welser2011a} or predict \cite{Danescu-niculescu-mizil2013} individuals' behavior.

%% DC 12: I like the _idea_ here of trying to explain why we see this kind of aggregate analysis.  This execution doesn't work so well for me, though.  It's not clear why a small community over the years, or snapshots, or pulled data are the reason people don't do analyses.  But, it did cause me to write the 
%In doing these analyses, it is quite common for researchers to have limited data in terms of time---one, or a few snapshots of the network are available\cite{}---or in terms of scope---only a small community is observed over the years \cite{Lewis2008}.
%% DC 12: This doesn't really help tell the story.
%It is also not unheard of cases in which datasets were made unavailable upon request from these social networks \footnote{\url{http://twitter.mpi-sws.org/}}.
%These increase the difficulty of scaling temporal models and limit the networks on which is possible to perform these kind of analysis --- Wikipedia is known for having their whole data available \cite{Panciera2009, Priedhorsky2007}. 

%% DC 12: Cites probably want to be put in here as they are found. 
These analyses often combine all available data into aggregate analyses of the entire community over its entire history.  This can be a natural response to limitations in the amount of available data: datasets may only capture a small part of the community's history \cite{}; timestamps may not be available \cite{}; snapshots may provide limited views of the community \cite{}; or the community itself might be small \cite{Lewis2008}.  Aggregate time-based analyses are also a natural first way to address these questions of community evolution.

In this paper, we argue it is likely that many of these aggregated views are misleading, because the conditions under which users join the community can vary greatly over time in ways that impact their behavior \cite{Miller2015}.  Among other things, the popularity, purpose, features, interface, and algorithms can change: Wikipedia circa 2005 and circa 2015 are very different, as are Facebook of 2005 and 2015.  Analyses---including some of our own past work---that fail to account for this change may miss important details of what's really going on.

We support this argument through an analysis of user effort in Reddit, one of the most popular and long-running online communities, based on a very large, recently released dataset of posting behavior \cite{}.  We address a number of questions commonly raised about users' effort in online communities: how often and how well do people post, what kinds of posts do they make, and where do they post them?  In each case, we compare aggregate analyses of posting behavior to ones that treat users and subcommunities in Reddit as yearly cohorts, and views that look at calendar time to views that normalize behavior based on the creation of a user or subcommunity.

We find that even this simple accounting for time reveals additional insights about Reddit beyond what commonly performed aggregate analyses show.  Users who join Reddit earlier post more and longer comments than those who join later.  We also see evidence of larger behavior changes in these older users over time in terms of posting versus commenting, changes we think are driven in part by the growth of the community.  Finally, we see that subcommunities created early in Reddit's history are likely to be much more active than those created later---and that the popularity of these communities spans across all user cohorts because of policy decisions Reddit makes in matching new users with popular subcommunities. 

We see this paper as both making specific contributions to understanding behavior in Reddit and more general contributions in showing the importance of considering change over time in analyzing online communities.  

%%Amit: write about key contributions. 