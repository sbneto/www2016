\section{Introduction}


Understanding the users evolution in a social netowork is essential for a variety of tasks: information diffusion, behavior predicion, recommendation tools, among other possibilities. Many works aim at explaining these temporal aspects of evolution. Some adopt a point of view of the whole network and try to understand the patterns of behavior inside of the network \ref{Zhu2014, Kooti2010}, some others adopt a more user focused point of view and try to assert more individuals characteristics \ref{Correa2010, Priedhorsky2007, Panciera2009} or predict their behavior \ref{Danescu-niculescu-mizil2013, }.

In dealing with these large temporal studies, it is quite common for researchers to have limited data in terms of time --- just a snapshot of the network is known \ref{} --- or limited in terms of scope --- only a small community is observerd over the years \ref{Lewis2008}. It is also not unheard of cases in which datasets were made unavailable upon request from these social networks \footnote{\url{http://twitter.mpi-sws.org/}}. These increase the difficulty of scaling temporal models and limit the networks on which is possible to perform these kind of analysis --- Wikipedia is known for having their whole data available \ref{Panciera2009, Priedhorsky2007}. A recent release of a large reddit dataset might help to improve this environment \ref{}.

These small pictures of the long term interaction of users with the social network can hide, bias or mislead our interpretation of their behavior and miss important information. Common averaging practices or careless selection of the sample might lead to wrong conclusions. In these work, we present some cases of how these can happen and simple cohorting approaches that help us to deal with these problems. We also show unexpected and paradoxal results related to the evolution of user effort (measured by number of characters written) in the reddit network.