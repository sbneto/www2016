
\section{Introduction}

Understanding the evolution of users in a social network is essential for a variety of tasks: monitoring community health, predicting individual user trajectories, and supporting effective recommendations, among others.  Many works aim at explaining these temporal aspects of evolution. Some adopt a point of view of the whole network and try to understand more general patterns of behavior \cite{Zhu2014, Kooti2010}, while others adopt a more user-centric point of view and try to model \cite{Correa2010, Priedhorsky2007, Panciera2009, Welser2011} or predict \cite{Danescu-niculescu-mizil2013} individuals' behavior.

These approaches often combine all available data into aggregate analyses of the whole community over its entire history.  This can be a natural response to limitations in the amount of available data: many datasets capture a small part of the community's history \cite{Artzi2012}; timestamps may not be available \cite{Priedhorsky2007, Pujol2010}; snapshots may provide limited views of the community \cite{Cosley2010}; or the community itself might be small \cite{Lewis2008}.  Aggregate time-based analyses are also a natural first way to address questions of community evolution.

\looseness=-1
However, we argue it is likely that many of these aggregated views are misleading. The conditions under which users join the community may vary greatly over time, and this might impact their behavior \cite{Miller2015}.  Among other things, the popularity, purpose, features, interface, and algorithms can change: Wikipedia circa 2005 and circa 2015 are very different, as are Facebook of 2005 and 2015.  Analysis---including some of our own past work---that fail to account for this change may miss important details of what is really going on.

We support this argument through an analysis of user effort in Reddit, one of the most popular and long-running online communities, based on a very large, recently released dataset of posting behavior.  We address a number of questions commonly raised about users' effort in online communities: how active are users, how hard do they work, and what kinds of things do they do?  In each case, we compare aggregate analyses of posting behavior to ones that treat users in Reddit as yearly cohorts, and views that focus on calendar time versus user-referential views that normalize behavior based on the creation date of a user.  We also look at differences within yearly cohorts, seeking differences between shorter and longer-lived users.

We find that even simple accountings for time reveal additional insights about Reddit beyond what commonly performed aggregate analyses can provide.  Users who join Reddit earlier post more and longer comments than those who join later, while users who survive longer start out both more active and more likely to comment than submit compared to users who leave Reddit early; none of these findings are obvious from commonly used analysis of user behavior.  

Further, we find that aggregate analysis can be downright misleading.  For instance, although average comment length decreases over time in an aggregate view, the comment length for surviving users increases over time in every cohort.  Likewise, an aggregate analysis suggests that longer-lived users post more over time; this is not the case.  Instead, users come into Reddit as active as they will ever be (somewhat akin to Panciera et al.'s finding that Wikipedians are ``born, not made'' \cite{Panciera2009}), and the rise in average activity for surviving users over time is driven fully by lower-activity users leaving early.     

We see this paper as both making specific contributions to understanding behavior in Reddit and a more general contribution around the importance of considering change over time in analyzing online communities. 


%% DC 12: I like the _idea_ here of trying to explain why we see this kind of aggregate analysis.  This execution doesn't work so well for me, though.  It's not clear why a small community over the years, or snapshots, or pulled data are the reason people don't do analyses.  But, it did cause me to write the 
%In doing these analyses, it is quite common for researchers to have limited data in terms of time---one, or a few snapshots of the network are available\cite{}---or in terms of scope---only a small community is observed over the years \cite{Lewis2008}.
%% DC 12: This doesn't really help tell the story.
%It is also not unheard of cases in which datasets were made unavailable upon request from these social networks \footnote{\url{http://twitter.mpi-sws.org/}}.
%These increase the difficulty of scaling temporal models and limit the networks on which is possible to perform these kind of analysis --- Wikipedia is known for having their whole data available \cite{Panciera2009, Priedhorsky2007}. 