%% DC 15: This tends to happen more often before discussion, so moving that there.	

%\section{Future Work}

%Although many observations were made regarding Reddit, the main goal of this paper is not to study Reddit in depth. Therefore, many observations still remain without an explanation. Being aware that latter cohort users decrease the size of their comments and tend post less allow us to question ourselves about the nature and motivation of these users. Are users commenting less because newer users they are less interested into engaging with other user or are they simply ``lazier'' and writing is becoming less attractive for the newer generations? If it is a matter of writing being a less preferred method of communication, could Reddit improve its users' interactions providing alternative ways to post media, such as pictures, audio and video? These are some questions that we can ask and needs further investigation of the data, and might evolve into new models of how user attention, effort and interaction in social networks are shifting through time.

%One interesting question that recurrently arises when we were performing survival analysis was regarding the users' being active or not. Potentially, users ``breaks'' from the network can influence our results. In the same way, ``death'' in a social network is not well defined --- users can delete their account, which is a clear sign of death, but they can also simple stop using the network. These questions of how to define active users and dead users and distinguishing patterns of behavior seems an interesting venue to pursue. Better definitions of ``active'' and ``dead'' users might allow us to characterize the burstiness of their behavior. Some users might only interact with the network in some specific occasions while some users might have a much more uniform pattern. Understanding how your network fare in terms of user burstiness is essential to understand how the users use the network and to set goals to improve or change the user experience. Also, a better definition of ``death'' would allow us to investigate the ``rebirth'' of users, that is, users that come back to the network. Whereas it can be considered a right censorship problem, it might pose a much more central issue, as network survival might not depend only in its ability to attract and retain users, but also in the ability to ``resurrect'' old users.

%% DC 10: This gets moved to discussion/future work, I think.
%% As with many analyses that focus on visible behavior, ours discounts lurkers despite their known importance as audience members \cite{} and future contrtibutors \cite{}.  Many lurkers likely vote, and thus may be even more important in a context like Reddit where votes affect content visibilty and provide explicit markers of attention and reputation.  However, the dataset does not have information on individual voters or timestamps, just the aggregate number of votes a post had received at the time of the crawl, making it impossible to effectively treat them as activity measures and ways to understand the behavior of those who voted.  The voting data might be much more useful, however, in questions that involve predicting a given user's future behavior based on whether and how other users respond to a user's early contributions \cite{joyce_kraut_2006, lampe_everything2, etc.}


