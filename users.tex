\section{Average posts per user}
In this section, we will use a common metric of user activity in online communities, the number of posts per user over time. Approaches that consider the total number of posts per user in a particular dataset \cite{Gruhl2004} and that analyzes the variation on the number of posts per user over the days \cite{Guo2009} have been applied to online social networks.

As we will see, both visualizing behavior relative to a user's creation time and using cohorts provide additional insight into posting activity in Reddit compared to a straightforward aggregate analysis based on calendar time .

\subsection{Calendar versus user-relative time}

We start with a common analysis used in this kind of work: aggregating behavior in the community based on calendar time.  Figure~\ref{fig:overall_posts}a shows the average number of posts per month by active users in that month.  Taken at face value, this 
suggests that over the first few years of Reddit, users became more active in posting, with per-user activity remaining more or less steady since mid-2011.

\looseness=-1

%% Sam 8: Reorganized the next 3 paragraphs, added information about throw away accounts
This average view hides several important aspects of users' activity dynamics. Previous work has looked into behavior relative to the user creation time. It has been shown that edge creation time in a social network relative to the user creation follows an exponential distribution \cite{Tomkins2008}. User lifetime, however, does not follow a exponential distribution and some types of user content generation follow a stretched exponential distribution \cite{Guo2009}. Throw-away accounts are one example of very short-lived users in Reddit \cite{Bergstrom2011}, for example. 

To address these characteristics, Figure~\ref{fig:overall_posts}b shows a different view that emphasizes the trajectory over a user's lifespan. We scale the x-axis not by clock time, as in Figure~\ref{fig:overall_posts}a, but by time since the user's first post: ``1'' on the x-axis refers to one year since the user's account first post, and so on. We call this the \textbf{time in the user referential}. One caution about interpreting graphs with time in the user referential is that the amount of data available rapidly decreases over time as users leave the community, meaning that values toward the right side of an individual data series are more subject to individual variation.  

%% Sam 11: Treating the misinterpretation as a hypothesis
%% Sam 12: The point of highlighting this as hypothesis is to point out at the end all the possible mistakes.
The evidence at this point supports the tempting hypothesis is that the longer a user survives, the more posts they make over time (\textbf{H1}).  This hypothesis, however, is incorrect; we will present a more nuanced description of what is happening informed by cohort-based analyses.

\subsection{New cohorts do not catch up}

\begin{figure*}[!tb]
\centering
\subimage[width=0.48, scale=0.42]{./images/avr_posts_per_user_over_time_cohorts.eps}
\subimage[width=0.48, scale=0.42]{./images/avr_posts_per_user_cohorts.eps}
\caption{Figure (a) shows the average number of posts per active users over clock time and Figure (b) the active users in the user-time referential, both segmented by users' cohorts. The user cohort is defined by the year of the user's creation time.  For comparison, the black line in Figure (a) represent the overall average.}
\label{fig:avr_posts_per_user_over_time_cohorts}
\end{figure*}

%% Sam 12: Adding the first research question here because we start cohorting here, and that is the main strategy of the paper.
Figure~\ref{fig:overall_posts}b suggests that older users are more active than newer ones, raising the question (\textbf{RQ1}): will new users
eventually follow in older users' footsteps?  Analyzing users' behavior by cohort is a reasonable way to address it.

Figure~\ref{fig:avr_posts_per_user_over_time_cohorts}a shows a first attempt at this analysis.  We can already observe a significant cohort effect: users from later cohorts appear to level off at significantly lower posting averages than users from earlier ones.  It suggests that newer users likely will never be as active as older ones on average. It also shows that surviving users are significantly more active than a naive average would suggest.

However, Figure~\ref{fig:avr_posts_per_user_over_time_cohorts}a also has an awkward anomaly: a rapid rise in the average number of posts during each cohort's first calendar year, especially in December. Combining cohort segmentation with user-referential analysis, as in Figure~\ref{fig:avr_posts_per_user_over_time_cohorts}b, helps smooth out this anomaly and aligns cohorts with each other.  Doing this alignment makes clear that differences between earlier and later cohorts are apparent early on.

\subsection{Does tenure predict activity, or vice versa?}

\begin{figure*}[!tb]
\centering
\subimage[width=0.31, scale=0.29]{./images/avr_posts_per_user_for_surviving_year_for_2010.eps}{2010 cohort}
\subimage[width=0.31, scale=0.29]{./images/avr_posts_per_user_for_surviving_year_for_2011.eps}{2011 cohort}
\subimage[width=0.31, scale=0.29]{./images/avr_posts_per_user_for_surviving_year_for_2012.eps}{2012 cohort}
\caption{Each Figure corresponds to one cohort, from 2010 to 2012, left to right. The users for each cohort are further divided in groups based on how long they survived: users that survived up to 1 year are labeled 0, from 1 to 2 years are labeled 1, and so on.  For all cohorts, longer-tenured users started at higher activity levels than shorter-tenured ones.}
\label{fig:avr_posts_per_user_for_surviving_year}
\end{figure*}

%% Sam 11: Accommodating the hypothesis, adding the second ``correct'' hypothesis
\looseness=-1
These graphs still support our initial hypothesis \textbf{H1} 
%the tempting conclusion that users become more active the longer they exist in Reddit, 
and they do not explain the rapid increase in posting activity in the first few months.  An alternative hypothesis, inspired by the ``Wikipedians are Born, not Made'' paper \cite{Panciera2009}, is that individual users come in with different posting propensities, and the rise over time is not that individual users become more active but that low-activity users leave the system (\textbf{H2}).  To examine this, we further segment each cohort by the number of years they were active in the system, as defined by the difference between their first and last post times.
 
Figure~\ref{fig:avr_posts_per_user_for_surviving_year} shows this analysis for the 2010, 2011 and 2012 cohorts\footnote{We only show these figures for the sake of saving space, but the same trends are observed in the other cohorts.}.  Across all cohorts and yearly survival sub-cohorts, users who leave earlier come in with a lower initial posting rate.  Thus, the rise in average posts per active user is driven by the fact that users who have high posting averages throughout their lifespan are the ones who are more likely to survive.  As the less active users leave the system, the average per active user increases.  In other words, the correct interpretation of Figure~\ref{fig:overall_posts}b isn't that longer-lived users post more.  It actually is that users who post more---right from the beginning---live longer. 

Combining Figure~\ref{fig:avr_posts_per_user_for_surviving_year}'s insight that the main reason why these curves increase is because the low posting users are dying sooner with the earlier observation that the stable activity level is lower for newer cohorts suggests that low-activity users from later cohorts tend to survive longer than those from earlier cohorts.  That is, people joining later in the community's life are less likely to be either committed users or leave than those from earlier on: they are more likely to be ``casual'' users that stick around.

\vspace{7pt} 
\section{Comment length}

\begin{figure*}[!tb]
\centering
\subimage[width=0.48, scale=0.42]{./images/avr_comment_size_over_time_cohorts.eps}
\subimage[width=0.48, scale=0.42]{./images/avr_comment_size_cohorts.eps}
\subimage[width=0.31, scale=0.29]{./images/avr_comment_length_for_surviving_year_for_2010.eps}{2010 cohort}
\subimage[width=0.31, scale=0.29]{./images/avr_comment_length_for_surviving_year_for_2011.eps}{2011 cohort}
\subimage[width=0.31, scale=0.29]{./images/avr_comment_length_for_surviving_year_for_2012.eps}{2012 cohort}
\caption{Figure (a) shows the average comment length over clock time and Figure (b) from the user-referential time. Both figures show the cohorted trends.  The overall average length per comment decreases over time, although for any individual cohort, it increases after a sharp initial drop. Figures (c), (d) and (e), similar to Figure~\ref{fig:avr_posts_per_user_for_surviving_year}, shows the monthly average comment length for active users in the cohorts of 2010, 2011 and 2012, segmented by the number of years that the user survived in the network.  Opposite the analysis for average posts, which showed that low-activity users were the first to leave Reddit, here, people who start out as longer commenters are \textit{more} likely to leave.}
\label{fig:comment_length}
\end{figure*}

%% Sam 12: Including the work in this section as a broad research question.
Activity as measured by the average number of posts per user is one proxy for user effort.  Comment length can also be considered as a proxy for user effort in the network.  Users that type more put more of their time in the network, contribute with more content, and might create stronger ties with the community. Thus, we put forward the following question (\textbf{RQ2}): how does comment length changed in the community over time, both overall and by cohort?

\subsection{Comment length drops over time}

%% Sam 11: Adding hypothesis tags. Later on to mention all the possible misleading hypothesis that we can come up with not considering cohorts and time
%% Sam 12: The point of highlighting this as hypothesis is to point out at the end all the possible mistakes.
Figure~\ref{fig:comment_length}a shows the overall comment length in Reddit over time (the darker line) and the overall length per cohort. 
Based on the downwards tendency of the overall comment length in Figure~\ref{fig:comment_length}a, one might hypothesize that users' commitment to the network is decreasing over time (\textbf{H3}), or that there is some community-wide norm toward shorter commenting (\textbf{H4}). 

However, this might not be the best way to interpret this information. Figure~\ref{fig:comment_length}b shows the comment length per cohort in the user referential time. An important observation here is that younger users start from a lower baseline comment length than older ones. Considering the fact that recent Reddit has experienced exponential growth, the weight when evaluating the overall average for Figures \ref{fig:comment_length}a and \ref{fig:comment_length}b is shifted towards the comment length for the ever-growing younger generation as the years go by; this younger generation brings down the average since their average is lower.

\subsection{Simpson's Paradox: the length also rises}

Let us go back to Figure~\ref{fig:comment_length}a, which shows the overall average comment length on Reddit over time. We see a clear trend towards declining length of comments in the overall line (the black line that averages across all users). This could be a warning sign for Reddit community managers, assuming longer comments are associated with more involved users and healthier discussions. A data analyst looking at these numbers might think about ways to promote longer comments on Reddit. 

However, in Figure~\ref{fig:comment_length}b, we saw that average comment length increases over time for every cohort. While later cohorts start at smaller comment length, after an initial drop, all cohorts show positive trends towards writing longer comments over time.  This is puzzling: when each of the cohorts exhibits a steady increase in their average comment length, how can the overall mean comment length decrease?  This anomaly is an instance of the Simpson's paradox \cite{simpson1951}, and occurs because we fail to properly condition on different cohorts when computing mean comment length. 

\begin{table}[!tb]
\centering
\tabcolsep=0.07cm
\singlespacing
\fontsize{9pt}{10.5pt}\selectfont
\begin{tabular}{|c|c|c|c|c|c|c|c|c|c|}
\cline{2-9}
\multicolumn{1}{c|}{} & \multicolumn{8}{c|}{Cohorts} \\ \hline
Year & 2007 & 2008 & 2009 & 2010 & 2011 & 2012 & 2013 & 2014 & Overall\\ \hline
2007 & 220 & - & - & - & - & - & - & - & 220 \\ \hline
2008 & 208 & 198 & - & - & - & - & - & - & 204 \\ \hline
2009 & 224 & 204 & 201 & - & - & - & - & - & 208 \\ \hline
2010 & 223 & 204 & 189 & 184 & - & - & - & - & 193 \\ \hline
2011 & 233 & 211 & 199 & 184 & 167 & - & - & - & 182 \\ \hline
2012 & 241 & 221 & 212 & 197 & 173 & 167 & - & - & 178 \\ \hline
2013 & 244 & 225 & 214 & 199 & 177 & 167 & 164 & - & 174 \\ \hline
2014 & 246 & 229 & 217 & 204 & 183 & 172 & 165 & 176 & 176 \\ \hline
\end{tabular}
\caption{Evolution of the average throughout the years for each cohort. Each column here is one cohort and each line is one year in time. Cohorts only start having data on the cohort year, therefore the upper diagonal is blank. On the right column we see the overall average for all users.}
\label{tab:simpson}
\end{table}

Table~\ref{tab:simpson} provides some clues to what might be going on. When we move down the rows, we observe an increasing tendency in each cohort column. It means that the average comment length increases for these users. However, when we move right through the columns, people in later cohorts tend to write less per comment. If we were to average each row, we would still get an overall increasing comment length per year, but that is not what we see in the overall column. What happens here is that the latter cohorts have many more users than earlier ones. Since their numbers increase year by year, we have a much larger contribution from them towards comments, compared to users of earlier cohorts. This uneven contribution leads to the paradox we observed in Figure~\ref{fig:comment_length}a. 

Without the decision to condition on cohorts, one would have gathered an entirely wrong conclusion. People are not writing less as they survive, rather those who tend to write less are joining the community in much larger numbers. Knowing this, one may focus on better onboarding processes for newcomers, or try to learn why users in later cohorts tend to write smaller comments on average.  

\begin{figure*}[!tb]
\centering
\subimage[width=0.48, scale=0.42]{./images/comments_per_submissions_over_time_cohorts.eps}
\subimage[width=0.48, scale=0.42]{./images/comments_per_submissions_cohorts.eps}
\subimage[width=0.23, scale=0.23]{./images/comments_per_submissions_for_surviving_year_for_2008.eps}{2008 cohort}
\subimage[width=0.23, scale=0.23]{./images/comments_per_submissions_for_surviving_year_for_2009.eps}{2009 cohort}
\subimage[width=0.23, scale=0.23]{./images/comments_per_submissions_for_surviving_year_for_2010.eps}{2010 cohort}
\subimage[width=0.23, scale=0.23]{./images/comments_per_submissions_for_surviving_year_for_2011.eps}{2011 cohort}
\caption{Figure (a) shows the average comment per submission ratio over clock time for the cohorts and the overall average. Figure (b) shows the average comment per submission from the user-referential time for the cohorts. Figures (c), (d), (e) and (f), similarly to Figure~\ref{fig:avr_posts_per_user_for_surviving_year}, shows the 2008, 2009, 2010, and 2011 cohorts, segmented by the number of years a user in the cohort survived.  As with average posts per month, users who stay active longer appear to start their careers with a relatively higher comments per submission ratio than users who abandon Reddit sooner.  Unlike that analysis, however, the early 2008 cohort ends up below the later cohorts in Figure (b).}
\label{fig:comments_submissions}
\end{figure*}

\subsection{New users burn brighter}
As with the posting per user, we can not say if the increase in the curves seen in \ref{fig:comment_length}b are due to the lower effort users dying first or because users are writing more as they live on the network. To answer this, \ref{fig:comment_length}c allow us to make two important observations: first, \textit{comment length does increase inside of each cohort}, no matter how long the user survives. Secondly, as a general trend, \textit{users that make longer comments inside of each cohort die faster}. This is quite surprising, given that we would expect people to put less effort when they are more likely to stop using the network.

\section{Kinds of contributions}

One common question from the literature is what sorts of activities users engage in, for instance, to categorize users into roles they play in the community\cite{Welser2011}. As with comments length, we propose the following research quetion (\textbf{RQ3}): how does users' activities changed in the community over time, both overall and by cohort?

\subsection{Over time, responsiveness increases}
Consider the case of Usenet: people who never start threads and only respond play the role of answerer, while there are other roles that include fostering discussion \cite{Welser2007}. These might naturally map onto people who primarily comment and who primarily submit in Reddit, respectively.  While submissions can be considered new content that an author generates, a comment can be considered as a contribution to an existing content from another author.

Since the total number of comments always surpasses the number of submissions, Figure~\ref{fig:comments_submissions}a shows the overall and cohorted evolution of the number of comments for each submission users made from 2008 to 2013. Here we see that users who most prefer commenting to submitting come from 2009 to 2011, and we observe that, over time, the average ratio of comments to submissions increases both overall and per-cohort for active users.

Again, we analyze our data from the user-time referential, as seen in Figure~\ref{fig:comments_submissions}b. It shows a clear pattern for users in earlier cohorts to have a lower comment per submission ratio than users in later cohorts ones, given that they both survived the same amount of time.  Surviving users from later cohorts also exhibit a more rapid increase in comments per submission than those from earlier cohorts.  In particular, the 2008 and 2009 cohorts increase much more slowly over time than those from 2010 onwards; later cohorts are more similar (although the 2012 and 2013 cohorts may level off lower than 2011 based on the limited data we have). 

\subsection{Comment early, comment often}

Figures~\ref{fig:comments_submissions}c-f shows the cohorts from 2008 to 2011 segmented by surviving year.  Three interesting observations arise from these data.  First, we see that just as in the analysis of average posts per user, the users who survive the longest in each cohort are also the ones who hit the ground running.  They start out with a high comment-to-submission ratio relative to users in their cohort who abandon Reddit more quickly.  This suggests that both the count of posts and the propensity to comment might be a strong predictor of user survival.

Second, and unlike the case for average post length, surviving users' behavior changes over time.  Figure~\ref{fig:avr_posts_per_user_for_surviving_year} shows that even for the most active users, they come in at a certain activity level and stay there, perhaps even slowly declining over time.  Here, in Figures~\ref{fig:comments_submissions}c-f, the ratio of comments to submissions increases over time; combined with the observation that overall activity stays steady, this suggests that the ratio is changing because people \textit{substitute} making their own submissions for commenting on others' posts.

Finally, this increase is most pronounced in the earlier cohorts of 2008 and 2009, with ratios more than doubling over their first year, much more than for later cohorts.
%% Sam 11: I don't quite remember why we wrote this, is it right? I can't really see how this is true from the figures.
%Still, the ratio for these earlier cohorts never rises to the level it does for surviving users from later cohorts. 
