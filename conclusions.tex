\section{Discussion}

In this section we discuss some of the processes that might explain our observations and refer to the associated literature. We're not taking a position on either of these as the mechanism that explains these results; both would be interesting avenues for future work.  We do suggest that looking at Reddit from a cohort and user-based view rather than an aggregate community view helped us uncover interesting phenomena and questions that would have been invisible to more commonly-performed analyses of community behavior. 

% Sam 12: Took if from the users' posting section to here. We should probably rethink this.
\subsection{Why more low activity users are surviving?}

%% DC 10: The status bit is intriguing but needs to be explained a little better.  Took a little shot at it but framing it in terms of cumulative advantage rather than status
%% DC 10:This, too, would need to be explained/justified/supported, and I can't come up with a plausible one, so deleting it.
%Yet another explanation would be that earlier users demographics were different in terms of age and interests, for example, and these correlate to the fact that they present a higher activity.
We have seen that users from latter cohorts have a lower posting average than in earlier cohorts. 
%% Sam 14: Found one talking about books in amazon, they call it a self-selection bias problem
%, in the same way that early ratings for a movie in a recommender system are likely to be higher than later ratings because the people who are most attuned to the movie are likely to see it earlier \cite{if_we_can_find_one}.
One plausible explanation is that users who find a community suffer from a self-selection problem: users that find reddit early in its life are also more likely than average to be those who will be attracted to it. Previous work has shown that online book reviews have a self-selection problem, since early reviews tend to be positively biased \cite{Li2008}. This would mean that the mixture of users joining in the early stage of the community are more likely to be the most active ones and the latter ones are more likely to be less active. A higher number of less active users joining the network would also account for their longer survival, although other mechanisms might also be in play.

Another is an argument based on cumulative advantage, status, and attention-seeking: surviving users from earlier cohorts might be more capable of producing content that gets attention from other users.  This would lead to them getting more comments and votes for their content, and people who get positive attention are more likely to return \cite{Halfaker2009, Choi2010, Sarkar2012}.  

\subsection{Why are comments getting shorter?}

One hypothesis that we might consider to explain the decreasing comment length of the users is associated to an ``initial value problem''. We can imagine that users, as they join the network, tend to produce content according to the norms of what they see \cite{Kooti2010, Danescu-niculescu-mizil2013}. The observed behavior of the comments length for the users in reddit is a initial drop, followed by steady increase as the user survives. If the starting point for the initial drops are taken as the average of the network, that is what is observed by the user in the network, the initial drop would place each cohort starting at lower levels than the previous one.

\subsection{Why comments per submissions are increasing?}

The majority of users in social networks are known to be lurkers: users that only seek information and passively observe, not engaging and contributing to the network \cite{Rafaeli2004, Nonnecke2000}. It is reasonable to expect the same from reddit. On the other hand, social networks often have a small number of ``power contributors'' \cite{Panciera2009, Kittur2007}.

When we consider the evolution of the number of comments per submission, we observer a decreasing trend for the older cohorts. Just as lurkers are the majority in the community and are attracted to it in search of information, we can imagine a set of users that are in the community in search of content and is whiling to engage in commenting, but hardly has the drive of a ``power contributor'' to bring new content into the network. Since in early reddit the amount of existing submissions was significant smaller than in the next few years, limited space existed for information seeking lurkers and ``commenters''. As the community grew, more content from an absolute point of view was present in the social network, therefore activities of information seek became more prominent and could explain the increase in commenting activity. 

%% Sam 15: Gone
%\subsection{Why the bias to 2008?}
%
% Sam 14: Still have to work on this one.
%Were we expecting different results? Defaults? Core consolidation?
%These observations allow us to conclude that, in the case of reddit, there are key subreddits that were created in 2008 that are the main focus of attention of all the users, although this is decreasing as new users join the network. Our hypothesis would not hold true in this case. This, however, might not hold true for other social networks, in which the communities or the content at the time at which users join the network might be their main focus of attention, highlighting again the importance of performing a cohort based analysis.

\section{Conclusions}

This work highlights the importance of taking time into consideration when analyzing users' evolution in social networks. We do so by cohorting the users based on their creation year. Although simple, this approach provides evidence of significant differences between a time-considering method with a naive overall averaging one. We also analyze the evolution of users and communities from a shifted time referential: considering the time of an action in relation to the user creation date. This also reveals unexpected behaviors that we would otherwise not notice.

From the user perspective, we found that user posting activity for surviving users is actually significant higher than a naive average. Also, controlling for survival, we found that the number of surviving low-posting users is increasing in the younger cohorts, low posting users die earlier than high posting users and users tend to decrease their posting activity as they survive.

Similarly, we analyzed user effort based on average comment length. We found that, while the overall average in the network seem to decrease, users actually write longer comments as they survive, independently of the survival time. %Cohorts, simpsons