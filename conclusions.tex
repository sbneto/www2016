\section{Discussion}

% Sam 12: Took if from the users' posting section to here. We should probably rethink this.
\subsection{What's going on?}

This raises interesting questions of why we see this behavior.  One plausible explanation is that users who find a community early in its life are also more likely than average to be those who will be attracted to it, in the same way that early ratings for a movie in a recommender system are likely to be higher than later ratings because the people who are most attuned to the movie are likely to see it earlier \cite{if_we_can_find_one}.  Another is an argument based on cumulative advantage, status, and attention-seeking: surviving users from earlier cohorts might be more capable of producing content that gets attention from other users.  This would lead to them getting more comments and votes for their content, and people who get positive attention are more likely to return \cite{joyce-kraut, wikipedia, everything2_papers}.  

We're not taking a position on either of these as the mechanism that explains these results; both would be interesting avenues for future work.  We do suggest that looking at Reddit from a cohort and user-based view rather than an aggregate community view helped us uncover interesting phenomena and questions that would have been invisible to more commonly-performed analyses of community behavior. 

\subsection{Why are comments getting shorter?}

Yet another hypothesis that we might consider is that users are lowering their activity due to an ``initial value problem''. We can imagine that users, as they join the network, they tend to produce content according to the norms of what they see. If we look at the cohort posting size over time superimposed with the average size for the whole network, we can see that the starting point of each cohort seems to agree to a reasonable extent to the average over the total network. This way, users would be simply reproducing things as they see in their early months, but as we have seen in Figure N, users start their life posting longer content, but there is a strong decrease in size for the early months before the size increases for the surviving users.



\section{Conclusions}

This work addresses some aspects of how to analyse the evolution of a social network and how to apply and avoid some pitfalls of cohort analysis. To do so, we analyse the reddit network and provide insights on the users posting behavior evolution, we identify a general tendency of newer users to write smaller comments and we discuss user survival from empirical standpoint. 

We also analyse subreddits evolution, considering the volume of activity, how commenting and submitting change in function of cohorts and the matter of community survival as subreddits can be seen as independent communities.
