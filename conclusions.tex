\section{Discussion}

In this section we discuss some of the processes that might explain our observations, and how they connect to other literature.  We're not arguing here that we know the answers; instead, we see these as interesting avenues for future work.  

\subsection{Why are newer ``active'' users less so?}

We have seen that users from later cohorts have a lower posting average than in earlier cohorts. 
One plausible explanation is that users self-select: users that find Reddit early in its life are also more likely than average to be those who will be attracted to it. Previous work has shown that online book reviews have a self-selection bias, where people who are more likely to like (or promote) the book review it earlier, leading to a positive early bias in an item's life \cite{Li2008}.  In Reddit's case, this would mean that the mixture of users joining in the early stage of the community would be disproportionately likely to be the most active ones and the latter ones are more likely to be less active. 

Another plausible hypothesis for later cohorts having a higher number of less active users could be that, over time, Reddit has accumulated an increasing number of valuable-but-small/niche communities.  The increased diversity might support a wider set of users in getting value, explaining the increased survival percentage.  The niche/smaller nature of newer communities might provide fewer opportunities to both submit and comment, explaining the lower average activity for surviving users. 

A third hypothesis is that Reddit overall is becoming more about consumption and voting on content rather than producing it.  Older users with contribution norms continue to contribute; newer users tend to provide audiences and feedback.  High-resolution voting data could be a real boon in understanding if this is true.

\subsection{Why are comments getting shorter?}

We also observed that overall, comment lengths are getting shorter over time.  

One hypothesis is that users are being shaped by an ``initial value problem''. We can imagine that users as they join the network, tend to produce content according to the norms of what they see \cite{Kooti2010, Danescu-niculescu-mizil2013}.  The observed behavior of the comments length for the users in Reddit is a initial drop, followed by steady increase as the user survives. If the starting point for the initial drops are taken as the average of the network, that is what is observed by the user in the network, the initial drop would place each cohort starting at lower levels than the previous one.  Figure~\ref{fig:comment_length}a presents some support this hypothesis: the initial month of each cohort year, which consists of data only from users who joined in that month, is quite close to the overall line from the prior month.  

Another hypothesis advanced by community members\footnote{See \url{https://www.reddit.com/r/TheoryOfReddit/comments/1a7aoj/retracing_the_evolution_of_reddit_through_post/  }.} is that Reddit's karma system favors shorter comments.  That is, people can get more upvotes for a given amount of effort by writing more, shorter comments.  This could be directly measured even with the available data, and might be the start of a very interesting line of future work that tries to model strategic posting and attention distribution behavior in Reddit. 

\subsection{Why do comments per submission increase?}

We also saw that comments per submission increase over time for surviving users, and that this is most dramatic for users who join earlier.

One process hypothesis is that this is because early in Reddit's life, there simply weren't as many submissions to comment on, meaning that people who wanted to be active contributors more or less had to submit in order to do so. 
As the community grew, more content became available, making information seeking more valuable---and perhaps increasing the value and ease of commenting. 

This question of ease and value might be more general, and tie to our observations about self-selection and karma accumulation.  Most users in social networks are known to be lurkers: only seeking information and passively observing, not engaging and contributing with content \cite{Rafaeli2004, Nonnecke2000}. Consumption in Reddit is valuable and easy, and some contributions are easier than others: reading is easier than voting; voting is easier than commenting; commenting is easier than submitting.  Only users for whom finding and submitting comments is relatively easy or relatively valuable are likely to be frequent submitters or ``power users'' \cite{Panciera2009, Kittur2007}. We suspect such users are more likely to be ones who found Reddit earlier, when it was relatively small, and stuck with it.

\subsection{Limitations and Future Work}

In this paper we focused our attention on behavior attributable to specific users, which in this dataset meant submissions and comments.  As with many analysis that focus on visible behavior, this means we miss important phenomena.  In particular, we discount lurkers despite their known importance as audience members \cite{Nonnecke2003} and potential future contributors \cite{Ridings2006}.  Many lurkers likely vote, and thus lurking may be even more important in a context like Reddit where votes affect content visibility and provide explicit markers of attention and reputation.  

However, the dataset does not have information on individual voters or timestamps, just the aggregate number of votes a post had received at the time of the crawl, making it impossible to effectively treat them as activity measures and ways to understand the behavior of those who voted.  The existing voting data might be much more useful, however, in addressing questions that involve predicting a given user's future behavior based on whether and how other users respond to a user's early contributions \cite{Joyce2006,Sarkar2012}.

Another blind spot that focusing on visible behavior can induce is our emphasis on active users.  This is a reasonable view of the community that focuses on what is happening, but our results should all be interpreted in the context of ``given the set of active users at any given time''.  Applying these results to questions that require considering all users would be a mistake.  

%% Sam 8: Added consideration about multi-user/account and deanonyzation
One of our assumptions was that one account is associated with one user. This might not be the case, as more than one user can share the same account \cite{Lampinen2014} or one user can have multiple accounts that are userd regularly or simply thrown away \cite{Bergstrom2011}. While it is not clear what is the impact of such behavior in our analyses, it raises the question of whether these users-accounts' associations can be de-anonymized.

We did, implicitly, consider survival in the analyses that broke cohort down by survival time; we see careful thinking about what it means to ``survive'' in a community as an interesting problem in its own right.  Potentially, users' ``breaks'' from the network can influence both our results and other analyses that assume users depart on their last visible day of activity. 
Focusing on activity also fails to account for actual deletion in many contexts.  In Reddit, activity from users is marked with a username of ``[deleted]'' (which we were able to ignore after realizing that one author had millions of comments!), but in some contexts, such as Wikipedia articles that are deleted, edit behavior on those articles do not show up in many data dumps.

\looseness=-1
These questions of how to define active users and dead users and distinguishing patterns of behavior seems an interesting venue to pursue. Better definitions of ``active'' and ``dead'' users might allow us to characterize the burstiness of their behavior.  Some users might only interact with the network in some specific occasions while some users might have a much more uniform pattern; in Wikipedia, the practice of leaving temporarily is so common it is called a ``wikibreak''. Understanding how your network fares in terms of user burstiness is essential to understand how the users use the network and to shape the user experience.  A better definition of ``death'' would allow us to investigate the ``rebirth'' of users, that is, users that come back to the community.  Rather than an annoying right censorship statistical problem, it might pose a much more central issue, as a community's survival might not depend only in its ability to attract and retain users, but also in the ability to ``resurrect'' old users. 

\section{Conclusions}

This work highlights the importance of taking time into consideration when analyzing users' evolution in social networks. We do so by cohorting the users based on their creation year. Although simple, this approach provides evidence of significant differences between methods that account for time with methods that do simple overall analyzes.  We also analyze the evolution of users and communities from a shifted time referential: considering the time of an action in relation to the user creation date. This also reveals unexpected phenomena that we would otherwise not notice.

%% Sam 12: Tying things with research questions and hypothesis
While investigating our research question \textbf{RQ1}, we found that user posting activity for surviving Reddit users is actually significantly higher than a naive average would suggest, that older users who survive are considerably more active than younger survivors, and that these newer users are unlikely to catch up.   Controlling for survival provided evidence for our hypothesis \textbf{H2}, supporting that users have a stable level of posting activity over time (with slightly decreasing patterns), the percentage of surviving but low-activity users is increasing in the younger cohorts and that low activity users dying faster is the main reason for the drop in the overall average curve.

%% Sam 12: Tying things with research questions and hypothesis
Similarly, when dealing with our research question \textbf{RQ2}, we analyzed user effort based on average comment length. We found that, while the overall average in Reddit seems to decrease, users actually write longer comments as they survive, no matter when they joined. Still, later cohorts of users that joined the network are writing smaller comments; their greater number leads to this version of Simpson's paradox, where where the overall average decreases while the series for each individual cohort increases. 

%% Sam 12: Tying things with research questions and hypothesis
Finally, we analyze the type of activities users engage in our research question \textbf{RQ3}, differentiating comments and submissions. 
We found that users with a higher comments per submission ratio are more likely to survive longer in the network. Even more, this behavior changes as the users survive---particularly for earlier cohorts. Users' comments per submissions patterns change, and their main mechanism to do so seems to be replacing their submitting by commenting behavior, as their posting activity remains stable.
%% Sam 11: Unclear if it is worth to mention things from the discussion here: they are not really conclusions, but more speculations on possible explanations. Also, since there is a bit of a space issue, I think this could go.
%We also discussed a possible explanation for this observation based on commenting being partially an information seeking task with an associated effort between lurking and submitting. This made it less likely in early Reddit, in which content was not as available as in later years.

%% Sam 8: Added one important take-away message in my opinion
%% Sam 11: Including some hypotesis
An important observation that we made is that the overall evolution of users' behavior in a network is driven by three factors: actual changes in users' behavior over time, users joining the network and users leaving the network. Failing to account for different demographics joining and leaving the network might limit our interpretation of the data (\textbf{H1}, \textbf{H3} or \textbf{H4}) and lead to wrong conclusions.

Both our and work and its limitations suggest fruitful directions for better understanding of users' evolution in both Reddit and online communities in general, directions we hope inspire other work in this area.  
