\section{Time matters} 
\subsection{Why accounting for time is important}

Communities grow and, with time, die. For any community, its users play a role in its evolution, but they are also simultaneously affected by the evolution of the community. Untangling this interplay can help make sense of patterns of activity in a community.

One useful way to understand the evolution of a community and its users is through time, as it provides a linear account of the growth (or decay) of overall activity, types of content, social norms, and structure of communities. To account for time, users on online communities are differentiated based on their age, such as when modeling their preferences \cite{McAuley2013} or analyzing the evolution of their language \cite{Danescu-niculescu-mizil2013}. These analyses uncover insights about the lifecycle of a user in a community: users' preferences and behavior change with their age in a community \cite{Panciera2010}, and their early experiences and activity shape future outcomes predictably \cite{Tan2015,Yang2009,Panciera2009, Miller2015}. 

However, much of past work on online communities ignores the time at which a user joins the community and analyzes all users together, irrespective of when they joined a community. 
This might be a mistake: communities may grow denser or sparser with time \cite{Leskovec2005}, develop new norms \cite{Kooti2010} and/or enact policies and rules guiding people's behavior \cite{Butler2008}.
These changes mean that people experience different versions of a community at different times, which can, in turn, affect their observed behavior. This interaction with the state of a community can confound conclusions about people's behavior, because the differences one observes may simply be due to changes in the community, rather than any significant change in the outcome variable of interest or the user population.  


%These  analyses normalize clock time to a user-relative time, such as measuring time since a user's first post in a community. 
%In this paper,we argue that gnoring the actual wall-clock time at which people joined can be a mistake. Changes in activity we might see between users might get washed out because of the internal differences due to their joining date, or may appear exaggerated when they are, in fact, no effects. As we will show, through a careful analysis of effect of a user's join cohort, not accounting for cohorts can lead us to absolutely wrong conclusions in some cases. 

\subsection{Cohorts are analytically useful}

To prevent such confounding, a common unit of analysis to control for such biases is cohort analysis, widely used in fields such as sociology \cite{Mason2012,Glenn2005}, economics \cite{Attanasio1993,Beldona2005}, and medicine \cite{Howartz1996,Davis2010}. A cohort is defined as a group of people who share a common characteristic, generally with respect to time. For example, people born in the same year, or those who joined a school at the same time, or got exposed to an intervention at similar times can be considered as cohorts. Such people in a cohort can be assumed to be exposed to the same state of the world and thus are more comparable to each other than people in other cohorts. 

For example, sociological studies often use students who join a school in the same year to understand the effect of interventions \cite{Goyette2008,Alexander2012}, and condition on the year in which people were born to understand people's  behavior, such as variations in financial decisions-making \cite{Attanasio1993} or opinions on issues \cite{Firebaugh1988,Jennings1996}. Similarly, medical studies interpret effects of drugs using cohorts of people with the same age group or lifelong exposure to correlated conditions \cite{Howartz1996,Davis2010}.  

Recent work shows that the importance of cohorts transfers to online communities as well. Just as people's behavior varies according to their biological age, their experience in an online community may vary with their age in the community and their year of joining. In Wikipedia, for example, we find substantial differences in the activities of cohorts of users who joined earlier versus those who joined later \cite{Welser2011}. Similarly, on review websites, users who join later tend to adopt different phrases than the older users who had joined earlier \cite{Danescu-niculescu-mizil2013}.

\subsection{What might cause these differences?}

These differences in activity between cohorts may be due to a number of reasons. It could be due to selection effects: people who are enthusiastic about a community or its goals are more likely to self-select as early members of a community, while others may be more likely to join later \cite{Li2008}. 

The norms in community may change over time, which could explain why users in later cohorts may behave differently. In many cases, it is a bottom-up process. Kooti et al. \cite{Kooti2010} showed that social conventions can define the evolution of a community and the early adopters play a major role in designing these conventions, even if at the time this is not known by them. Examples include adoption of `RT', a retweeting norm by Twitter users and the subsequent introduction of the Retweet button on Twitter \cite{Kooti2010}; change in language use by new and old users on review websites \cite{Danescu-niculescu-mizil2013}; and assumptions of clear roles and responsibilities on Wikipedia \cite{Kittur2007a}. In other cases, it may be directed by the community managers. For instance, the makers of Digg unilaterally changed the nature of the community by introducing a new version of the website, leading to a sudden change in norms and behavior in the community \cite{Ingram2014,Lardinois2014}. 

The growth of a community may also affect people's behavior. Successful communities often grow very rapidly, which can be both good and bad for people's experience with the community. On one hand, growth would imply availability of a larger chunk of content to choose from. On the other, it might be harder to connect to others and get responses in a bigger community. A community may also need to adopt new rules and policies to manage growth and newcomers, as in the evolution of Wikipedia \cite{Choi2010,Bryant2005}, and in those cases, the experience of later cohorts of users may be vastly different from the initial ones who joined before formal rules were in place. 

Finally, patterns of use may change because the overall population of Internet users is still changing. As more and different people become connected with the web, their influx may lead to observed change in activity patterns.  This also affects technology use: people who did not grow up in a technological environment differ in their social media and search usage compared to younger generations\cite{Correa2010,Beldona2005}. 

All of the above reasons suggest that users from different cohorts are likely to be different, which has also been demonstrated in online and offline communities \cite{Ryder1965,Danescu-niculescu-mizil2013,Prensky2001,Correa2010}. Accounting for these differences can be helpful for making conclusions about outcomes of interest, such as user's activity levels, their survival, among many other possibilities.



 



%\begin{itemize}
    %\item The Taste for Privacy: An Analysis of College Student Privacy Settings in an Online Social Network \cite{Lewis2008}: Studies which characteristics are predictive of whether or not users are going to set their profile as public or private in Facebook. Raises questions about the limitations of the work because data collected came from a single cohort of users in a college.
%
    %\item Social selection and peer influence in an online social network \cite{Lewis2012a}: Yet another study based on a single cohort of Facebook data for college students. Discuss the relationship between homophily in creating connections and influence over the course of a connection.
%
    %\item Who interacts on the Web?: The intersection of users' personality and social media use \cite{Correa2010}: Studies how personality traits correlate with social media usage controlling for demographic variables age, gender, race, education and income. One of the research questions was whether user age cohorts influence social media usage. They found significant correlation of some personality traits with social media usage for the younger cohort (users from 18 to 29). They also acknowledge the lack of research on how age influences interaction on social media, pointing out that significant differences emerge from people that grew on a digital environment when compared to the ones that were introduced to the technology at a later time.
%
    %\item ``I LOVE THIS SITE!'' vs. ``It's a little girly'': Perceptions of and Initial User Experience with Pinterest \cite{Miller2015}: Initial experience matters! 
%
    %\item No Country for Old Members : User Lifecycle and Linguistic Change in Online Communities \cite{Danescu-niculescu-mizil2013}: User experience changes their behavior over time but they also come with some linguistic predispositions.
%
    %\item All Who Wander : On the Prevalence and Characteristics of Multi-community Engagement \cite{Tan2015}: Survival does depend on user initial activities.
%
    %\item Wikipedians are Born, Not Made \cite{Panciera2009}: Users do have predispositions. Does that mean they do not change and we are simply sampling differently?
%
    %\item Creating , Destroying , and Restoring Value in Wikipedia \cite{Priedhorsky2007}: Not clear where it fits.
%
    %\item The Impact of Membership Overlap on the Survival of Online Communities \cite{Zhu2014}: The survival of communities depends on the type of users that participate in it, and sharing certain types of users --- core members from other communities that are not core members in the focal community --- can be beneficial for community survival. Also, concepts of young and mature communities play a important role when analyzing community activity level, where young communities benefit from sharing members from matures communities. 
%
    %\item No Country for Old Members : User Lifecycle and Linguistic Change in Online Communities: Highlights the interplay on community language change and user adoption of new norms. As a general pattern, newcomers start learning the norms of the community and, as they age, they become more conservative in adopting new norms. Users that are more flexible in assimilating new norms have a higher survival rate.
    %
%\end{itemize}


