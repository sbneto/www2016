\section{Previous Work}

The Taste for Privacy: An Analysis of College Student Privacy Settings in an Online Social Network \cite{Lewis2008}: Studies which characteristics are predictive of whether or not users are going to set their profile as public or private in Facebook. Raises questions about the limitations of the work because data collected came from a single cohort of users in a college.

Social selection and peer influence in an online social network \cite{Lewis2012a}: Yet another study based on a single cohort of Facebook data for college students. Discuss the relationship between homophily in creating connections and influence over the course of a connection.

Who interacts on the Web?: The intersection of users' personality and social media use \cite{Correa2010}: Studies how personality traits correlate with social media usage controlling for demographic variables age, gender, race, education and income. One of the research questions was whether user age cohorts influence social media usage. They found significant correlation of some personality traits with social media usage for the younger cohort (users from 18 to 29). They also acknowledge the lack of research on how age influences interaction on social media, pointing out that significant differences emerge from people that grew on a digital environment when compared to the ones that were introduced to the technology at a later time.

``I LOVE THIS SITE!'' vs. ``It's a little girly'': Perceptions of and Initial User Experience with Pinterest \cite{Miller2015}: Initial experience matters! 

No Country for Old Members : User Lifecycle and Linguistic Change in Online Communities \cite{Danescu-niculescu-mizil2013}: User experience changes their behavior over time but they also come with some linguistic predispositions.

All Who Wander : On the Prevalence and Characteristics of Multi-community Engagement \cite{Tan2015}: Survival does depend on user initial activities.

Wikipedians are Born, Not Made \cite{Panciera2009}: Users do have predispositions. Does that mean they do not change and we are simply sampling differently?

Creating , Destroying , and Restoring Value in Wikipedia \cite{Priedhorsky2007}: Not clear where it fits.

The Impact of Membership Overlap on the Survival of Online Communities \cite{Zhu2014}: I have to read again.
